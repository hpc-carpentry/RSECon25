\begin{frame}{Where does HPC Carpentry fit?}
	\begin{enumerate}
		\item Knowing about Git and functions does not quite solve all problems.
		\item Researchers with HPC requirements do not necessarily know how to effectively scale their applications
		\item HPC facility operators become frustrated by non-knowledgable users making use of shared resources.
		\item Candidate solution: Use Carpentries techniques.
	\end{enumerate}
		\note[item]{So once you write using functions and your code is in GitHub, you try running it on an HPC and you might find it runs a bit faster because it just so happens that the login node has a better spec than your average laptop or desktop. But people start yelling at you for running stuff on the login node and you don't know what the problem is because you've never heard about the Slurm scheduler thing and you neither does MPI sound familiar}
	\note[item]{Using the Carpentries' style and pedagogy we HPC Carpentry address this problem}
\end{frame}

\begin{frame}{How did it all begin}
	\begin{itemize}
		\item Earliest commit 2013
		\item Peter Steinbach blog post (2017) - HPC in a day
		\item CarpentryCon 2018, 2020, 2022
		\item Super Computing BoF 17, 18, 19, 21
	\end{itemize}
	\note[item]{The earliest commit I could find in the hpc-carpentry organisation was 2013. I watched a recording of a presentation given by Andrew Reid at SIGHPC about a year ago to get an idea of the history of HPC Carpentry and these were some of the events that Andrew mentioned that were significant in the spread of the word of HPC Carpentry, by talking to people, running workshops and getting feedback.}
\end{frame}



\begin{frame} {Where are we now?}
	\begin{enumerate}
		\item We are doing things In the Carpentries Way
		\begin{itemize}
			\item two day setting
			\item get feedback
			\item students type along with instructors
		\end{itemize}
		\item two main takeaways:
			\begin{itemize}
				\item muscle memory 
				\item vocabulary
			\end{itemize}
		\item not experts after two days but:
			\begin{itemize}
				\item know what answer looks like
				\item know how to find answers
			\end{itemize} 
	\end{enumerate}
	

	\note[item]{}
\end{frame}