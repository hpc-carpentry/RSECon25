\begin{frame}{Where does HPC Carpentry fit?}
	\begin{enumerate}
		\item Knowing about Git and functions does not quite solve all problems
		\item Like SC researchers amenable with using HPC but curriculum doesn't prepare them for running stuff on a cluster. We can fix this using carpentry style pedagogy.
		teach
		\begin{itemize}
			\item two day setting
			\item get feedback
			\item students type along with instructors
		\end{itemize}
		\item two main takeaways:
			\begin{itemize}
				\item muscle memory 
				\item vocabulary
			\end{itemize}
		\item not experts after two days but:
			\begin{itemize}
				\item know what answer looks like
				\item know how to find answers
			\end{itemize} 
	\end{enumerate}
	
	\note[item]{So once you write using functions and your code is in GitHub, you try running it on an HPC and you might find it runs a bit faster because it just so happens that the login node has a better spec than your average laptop or desktop. But people start yelling at you for running stuff on the login node and you don't know what the problem is because you've never heard about the Slurm scheduler thing and you neither does MPI sound familiar}
	\note[item]{Using the Carpentries' style and pedagogy we HPC Carpentry address this problem}
\end{frame}